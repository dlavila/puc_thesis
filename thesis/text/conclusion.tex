\chapter{CONCLUSION}
\label{chapter:conclusion}
\section{Summary}

This thesis deals with the use of discrete-time filters to lower the noise present in the detectors front-end circuits. Main contributions of this work are: the development of a new mathematical framework a design-oriented analysis of discrete-time filters in the discrete-time domain; and the design and implementation of a switched-capacitor filter for arbitrary weighting function synthesis to be included in the Bean 2 IC.

The noise analysis methodology introduced in this work allows to easily compute the optimal discrete-time filter that maximize the SNR of a typical detector front-end circuit, a difficult task to undertake using the traditional methods for continuous-time networks.

The SC filter for arbitrary weighting function synthesis for the Bean 2 IC has been also presented, along with post-layout simulations for functionality verification. The use of this filter,  along with a proper characterization of the CSA and detector noise statistics, will allow to design an optimum filter based on the noise analysis presented in this work.



%Finally, a xew fully-differential PRS ADC architecture was introduced, along with its noise and non-idealities analysis, simulations and test results. This converter features a ca- pacitance spread of one, a very small area, a low-power consumption and a reconfigurable resolution.
%In this work, a sub-optimal microelectronic design flow was achieved, regardless of the problems introduced by the use of the manufacturer libraries for the schematic and layout editor used. A typical design flow comprises the schematic design, schematic simulations, layout edition, parasitics extraction and post-layout simulations including the extracted parasitics components. Due to the problems introduced by the use of the manufacturer libraries, the parasitics extraction and post-layout simulation could not be completed. Also, because of some inconsistencies in the schematic simulator used, the

\section{Future work}