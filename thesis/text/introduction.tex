\chapter{INTRODUCTION}
\label{chapter:introduction}
\section{Particle physics experiments}
Particle physics, also called High Energy Physics, is the branch of physics that studies the fundamental constituents of matter and radiation, and their mutual interactions. It aims to answer some of the profound questions of physics, with benefits spanning everything from advancing humankind’s understanding of the universe, to applications in other fields of science as well as daily life \citep{tuttle101}.

The main tools used by experimental particle physicists are particle accelerators, devices that uses electromagnetic fields to accelerate charged particles to relativistic speeds and to contain them in  well-defined beams \citep{livingston1962particle}. This beams of are used to generate particle collisions, which can be of a single beam against a stationary target or two beams directed against each other, that's why in the context of particle physics particle accelerators are best known as colliders. 

%The main tools used by experimental particle physicists are particle accelerators, which uses electromagnetic fields to accelerate charged particles to relativistic speeds and to contain them in well-defined beams \citep{livingston1962particle}. 

%Among the different kind of accelerators, colliders are the common variation used for high energy physics, this devices collide a single beam against a stationary target or two beams directed against each other.

Explicar por que es necesario crear colisionadores mas grandes y sofisticados.\\

\section{Electronics for particle physics experiments}

Explicar lo que es un detector de partículas y hablar de el front-end de un detector, 


\section{Noise minimization in circuits for particle physics instrumentation}

Explicar que el ruido limita la resolución de un detector, explicar de las bases del análisis de ruido, explicar los avances, x ejemplo las nuevas técnicas para minimizar el ruido del amplificador de carga, explicar los nuevos usos y la introducción de filtros discretos.

\section{Thesis content}
Chapter 2 starts with an introduction to the project that prompt the work of this thesis, the design and implementation of a second iteration of The Bean, an instrumentation ASIC which forms part of the proposal for a new particle accelerator, The International Linear Collider (ILC). Its followed by an overview of the motivations that lead to the development of a new mathematical framework for noise analysis in discrete-time filters, alongside with the presentation of the requeriments for a filter intended to take full advantage of this framework.
Chapter 3 presents the complete formulation of this noise analysis, including examples and applications for optimal filter computation. In Chapter 4 , the design and implementation of a filter for arbitrary weighting function synthesis are presented. Chapter 5 . Finally, Chapter 6 summarizes the 


Chapter 6 summarizes the conclusions 

 in the \mbox{discrete-time} domain.

intended to be used in the next 

the design and implementation of an instrumentation ASIC 

%Although the work presented in this thesis is part of the design and implementation of a specific instrumentation ASIC for particle physics, it aims to develop the tools to effectively integrate discrete-time filters as a practical solutions to the noise minimization problem in a typical detector readout scheme. 

to 

Main work of this thesis 

Chapter 2 includes an overview of the detector system in which 